\documentclass{article}
\usepackage{listings}

\begin{document}

\section*{1.5.1}

To be demonstrated:

$$ (x<0 \land y=2) \Rightarrow A \land A \Rightarrow (x<-2 \land y>0) $$

Because of the transitivity of implication it must therefore hold:

$$ (x<0 \land y=2) \Rightarrow (x<-2 \land y>0) $$

Lemma: $ A \Rightarrow (B \land C) \equiv (A \Rightarrow B) \land (A \Rightarrow C) $

Proof:

$$ A \Rightarrow (B \land C) $$
$$ \neg A \lor (B \land C) $$
$$ (\neg A \lor B) \land (\neg A \lor C) $$
$$ (A \Rightarrow B) \land (A \Rightarrow C) $$

It must therefore hold that $x<0 \Rightarrow x<-2$ and $y=2 \Rightarrow y>0$. The first
is not the case when x=-1.

Therefore, no such A can exist.

\section*{1.5.2}

To be demonstrated:

$$ \not\exists A,B: (x>2 \Rightarrow A) \land (A \Rightarrow B \land x>2) \land (B \Rightarrow x>5) $$

By the transitivity of implication it must hold that

$$ x>2 \Rightarrow x>2 $$

This is false (e.g.: x=3), and therefore there can be no such A, B.

\section*{1.5.3}

To be shown:

$$ (B \lor C) \Rightarrow ((A \land B) \Rightarrow C \equiv A \land \neg C \Rightarrow \neg B $$

Proof:

Definition of implication:

$$ (\neg B \land \neg C) \lor ((\neg A \lor \neg B) \lor C) \equiv \neg A \lor C \lor \neg B $$

DeMorgan:

$$ (\neg B \land \neg C ) \lor \neg A \lor \neg B \lor C \equiv \neg A \lor \neg B \lor C $$

Absorption:

$$ \neg A \lor \neg B \lor C \equiv \neg A \lor \neg B \lor C $$

These statements are trivially equivalent.

\section*{1.5.4}

We have to find a mapping so that the expression can evaluate to false.

$$ (((x<0 \Rightarrow A) \land (\neg A \Rightarrow x<-2)) \Rightarrow (y=3 \Rightarrow A)) \land (x<-2 \Rightarrow x<0) $$
$$ ((\neg (x<0 \Rightarrow A) \lor \neg (\neg A \Rightarrow x<-2)) \lor (y=3 \Rightarrow A)) \land (x<-2 \Rightarrow x<0)  $$
$$ ((\neg (\neg x<0 \lor A) \lor \neg (A \lor x<-2)) \lor (\neg y=3 \lor A)) \land (x<-1 \Rightarrow x<0) $$
$$ ((x<0 \land \neg A) \lor (\neg A \land \neg x<-2) \lor (\neg y=3 \lor A)) \land (x<-2 \Rightarrow x<0)  $$
$$ ((x<0 \land \neg A) \lor (\neg A \land \neg x<-2) \lor (\neg y=3 \lor A)) \land (\neg x<-2 \lor x<0)  $$
$$ (x<0 \land \neg A \land (\neg x<-2 \lor x<0)) \lor (\neg A \land \neg x<-2 \land (\neg x<-2 \lor x<0)) \lor ((\neg
y=3 \lor A) \land \neg x <-2 \land x<0) $$
$$ (x<0 \land \neg A \land \neg x<-2) \lor (\neg A \land \neg x<-2 \land x<0) \lor (\neg y=3 \land \neg x<-2 \land x<0) \lor (\neg y=3 \land \neg x<-2 \land x<0) $$
$$ (x<0 \land \neg A \land \neg x<-2) \lor (\neg y=3 \land \neg x<-2 \land x<0) $$

This is not true for the case $A$, $y = 3$.

\section*{1.6}

$$ A:\equiv true $$
$$ B:\equiv \forall n:true $$
$$ C:\equiv \forall n:true \land \forall m:true $$
$$ D:\equiv \forall n:true \land \forall m:true \land m \geq 0 $$
$$ E:\equiv \forall n:true \land \forall m:true \land m \geq 0 \land p = m $$
$$ F:\equiv \forall n:true \land \forall m:true \land m<0 $$
$$ G:\equiv \forall n:true \land \forall m:true \land m<0 \land p=-m $$
$$ H:\equiv \forall n:true \land \forall m:true \land (p=m \lor p=-m) $$
$$ I:\equiv \forall n:true \land \forall m:true \land (p=m \lor p=-m) \land i=0 $$
$$ J:\equiv \forall n:true \land \forall m:true \land (p=m \lor p=-m) \land i=0 \land x=n $$
$$ K:\equiv \forall n:true \land \forall m:true \land (p=m \lor p=-m) \land i \leq p \land p \geq 0 \land x=n^{i+1} $$
$$ L:\equiv \forall n:true \land \forall m:true \land (p=m \lor p=-m) \land i<p \land p \geq 0 \land x=n^{i+1} $$
$$ M:\equiv \forall n:true \land \forall m:true \land (p=m \lor p=-m) \land i<p \land p \geq 0 \land x=n^{i+2} $$
$$ N:\equiv \forall n:true \land \forall m:true \land (p=m \lor p=-m) \land i \leq p \land p \geq 0 \land x=n^{i+1} $$
$$ O:\equiv \forall n:true \land \forall m:true \land (p=m \lor p=-m) \land i=p \land p \geq 0 \land x=n^{i+1} $$

\section*{1.7}

\begin{lstlisting}
x=0;
n=read();
if(n>=0)
{
        i=0;
        x=1;
        while(i<n)
        {
                x=i*x;
                i=i+1;
        }
}
\end{lstlisting}

\section*{1.8}

\begin{lstlisting}
a=read();
b=read();

if(a>b)
{
        p=-1;
}
else
{
        if(a>=0)
        {
                bf=1;
                i=0;
                while(i<b)
                {
                        bf=i*bf;
                        i=i+1;
                }
                af=1;
                i=0;
                while(i<a)
                {
                        af=i*af;
                        i=i+1;
                }
                ks=0;
                k=a;
                while(k<b)
                {
                        ks=ks+k;
                        k=k+1;
                }
                p=bf/af*ks;
        }
        else
        {
                ks=0;
                k=0;
                while(k<b)
                {
                        ks=ks+k;
                        k=k+1;
                }
                p=ks;
        }
}
\end{lstlisting}

\end{document}
